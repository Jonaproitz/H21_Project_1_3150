\documentclass[english,notitlepage]{revtex4-1}  % defines the basic parameters of the document
%For preview: skriv i terminal: latexmk -pdf -pvc filnavn



% if you want a single-column, remove reprint

% allows special characters (including æøå)
\usepackage[utf8]{inputenc}
%\usepackage[english]{babel}

%% note that you may need to download some of these packages manually, it depends on your setup.
%% I recommend downloading TeXMaker, because it includes a large library of the most common packages.

\usepackage{physics,amssymb}  % mathematical symbols (physics imports amsmath)
\include{amsmath}
\usepackage{graphicx}         % include graphics such as plots
\usepackage{xcolor}           % set colors
\usepackage{hyperref}         % automagic cross-referencing (this is GODLIKE)
\usepackage{listings}         % display code
\usepackage{subfigure}        % imports a lot of cool and useful figure commands
\usepackage{float}
%\usepackage[section]{placeins}
\usepackage{algorithm}
\usepackage[noend]{algpseudocode}
\usepackage{subfigure}
\usepackage{tikz}
\usetikzlibrary{quantikz}
% defines the color of hyperref objects
% Blending two colors:  blue!80!black  =  80% blue and 20% black
\hypersetup{ % this is just my personal choice, feel free to change things
    colorlinks,
    linkcolor={red!50!black},
    citecolor={blue!50!black},
    urlcolor={blue!80!black}}

%% Defines the style of the programming listing
%% This is actually my personal template, go ahead and change stuff if you want



%% USEFUL LINKS:
%%
%%   UiO LaTeX guides:        https://www.mn.uio.no/ifi/tjenester/it/hjelp/latex/
%%   mathematics:             https://en.wikibooks.org/wiki/LaTeX/Mathematics

%%   PHYSICS !                https://mirror.hmc.edu/ctan/macros/latex/contrib/physics/physics.pdf

%%   the basics of Tikz:       https://en.wikibooks.org/wiki/LaTeX/PGF/Tikz
%%   all the colors!:          https://en.wikibooks.org/wiki/LaTeX/Colors
%%   how to draw tables:       https://en.wikibooks.org/wiki/LaTeX/Tables
%%   code listing styles:      https://en.wikibooks.org/wiki/LaTeX/Source_Code_Listings
%%   \includegraphics          https://en.wikibooks.org/wiki/LaTeX/Importing_Graphics
%%   learn more about figures  https://en.wikibooks.org/wiki/LaTeX/Floats,_Figures_and_Captions
%%   automagic bibliography:   https://en.wikibooks.org/wiki/LaTeX/Bibliography_Management  (this one is kinda difficult the first time)
%%   REVTeX Guide:             http://www.physics.csbsju.edu/370/papers/Journal_Style_Manuals/auguide4-1.pdf
%%
%%   (this document is of class "revtex4-1", the REVTeX Guide explains how the class works)


%% CREATING THE .pdf FILE USING LINUX IN THE TERMINAL
%%
%% [terminal]$ pdflatex template.tex
%%
%% Run the command twice, always.
%% If you want to use \footnote, you need to run these commands (IN THIS SPECIFIC ORDER)
%%
%% [terminal]$ pdflatex template.tex
%% [terminal]$ bibtex template
%% [terminal]$ pdflatex template.tex
%% [terminal]$ pdflatex template.tex
%%
%% Don't ask me why, I don't know.

\begin{document}

\title{FYS3150 oppgavesett 1}

\author{Jon Aleksander Prøitz og Marius Torsvoll}

\begin{abstract}
\noindent Alle løsninger på programmeringsspørsmålene kan finnes her: \\
\url{https://github.com/Jonaproitz/Project_1}
\end{abstract}

\maketitle
%\date{Received - / Accepted -}


\section*{Oppgave 1.}
    \label{sec:oppgave1}
    Vis at:
    \begin{equation*}
            u(x) 
        =   1-(1-e^{-10})x - e^{-10x}
    \end{equation*}
    \begin{equation*}
            -\frac{\partial^2 u}{\partial x^2} 
        =   100e^{-10x}
    \end{equation*}
    Dette sjekkes med en kjapp derivasjon
    \begin{equation*}
            \left(1-(1-e^{-10})x - e^{-10x}\right)'' 
        \Rightarrow 
            \left((1-e^{-10}) + 10e^{-10x} \right)' 
        =   -100e^{-10x}
        =   -f(x)
    \end{equation*}


\section*{Oppgave 2.}
    \label{sec:oppgave2}
    Se githublink 


\section*{Oppgave 3.}
    \begin{equation*}
            -\frac{d ^2 u}{dx^2} 
        =   \frac{u(x-h)+2u(x)- u(x+h)}{h^2} + O(h^2)
    \end{equation*}
    Diskretiserer den dobbeltderiverte med følgende definisjoner.
    \begin{align*}
            x 
        \rightarrow 
            x_0, x_1, x_2 ....\\
            u(x) 
        \rightarrow 
            u_0, u_1, u_2 ....\\
            f(x) 
        \rightarrow 
            f_0, f_1 , f_2 ....
    \end{align*}
    Vi tilnærmer $u \approx v$ og får:
    \begin{equation*}
            -\frac{d^2 v_i}{dx^2} 
        =   -v_{i-1} + 2v_i - v_{i+1} 
        =   f_ih^2
    \end{equation*}


\section*{Oppgave 4.}
    \label{sec:oppgave4}
    Settet med likninger fra oppgave 3 kan skrives som: 
    \begin{align*}
            2v_i - v_2 
        =   h^2 f_1  + v_0
        =   g_1\\
            -v_1 + 2v_2 -v_3 
        =   h^2 f_2 
        =   g_2\\
            -v_2 + 2v_3- v_4 
        =   h^2 f_3 
        =   g_3\\
        \vdots\\
            -v_{m-2} + 2 v_{m-1} 
        =   h^2 f_{m-1} + v_m 
        =   g_{m-1}
    \end{align*}
    Som kan skrives om til:
    \begin{equation*}
            \begin{pmatrix}
                2 & -1 & 0 &0 & 0 & \dots
                \\
                -1 & 2 & -1 & 0 & 0 & \dots
                \\
                0 & -1 & 2 & -1 & 0 & \dots
                \\
                & & & &  \ddots
                \\
                0 & 0 & 0 & 0 & -1 & 2 
            \end{pmatrix}
            \begin{pmatrix}
                _1 \\ v_2 \\ v_3 \\ \vdots \\ v_n 
            \end{pmatrix} 
        = 
            \begin{pmatrix}
                g_1 \\ g_2 \\ g_3 \\ \vdots \\ g_n
            \end{pmatrix}
    \end{equation*}


\section*{Oppgave 5}
    \subsection*{a}
        \label{sec:oppgave5a}
        Siden første og siste element er kjente verdier vil da  $n = m-2$
    \subsection*{b}
        \label{sec:oppgave5b}

\begin{acknowledgements}
Vi ønsker å takke det norske folk for deres støtte og hjelp i dette krevende arbeidet. 
\end{acknowledgements}

   
\end{document}
