\documentclass[english,notitlepage]{revtex4-1}  % defines the basic parameters of the document
%For preview: skriv i terminal: latexmk -pdf -pvc filnavn



% if you want a single-column, remove reprint

% allows special characters (including æøå)
\usepackage[utf8]{inputenc}
%\usepackage[english]{babel}

%% note that you may need to download some of these packages manually, it depends on your setup.
%% I recommend downloading TeXMaker, because it includes a large library of the most common packages.

\usepackage{physics,amssymb}  % mathematical symbols (physics imports amsmath)
\include{amsmath}
\usepackage{graphicx}         % include graphics such as plots
\usepackage{xcolor}           % set colors
\usepackage{hyperref}         % automagic cross-referencing (this is GODLIKE)
\usepackage{listings}         % display code
\usepackage{subfigure}        % imports a lot of cool and useful figure commands
\usepackage{float}
%\usepackage[section]{placeins}
\usepackage{algorithm}
\usepackage[noend]{algpseudocode}
\usepackage{subfigure}
\usepackage{tikz}
\usetikzlibrary{quantikz}
% defines the color of hyperref objects
% Blending two colors:  blue!80!black  =  80% blue and 20% black
\hypersetup{ % this is just my personal choice, feel free to change things
    colorlinks,
    linkcolor={red!50!black},
    citecolor={blue!50!black},
    urlcolor={blue!80!black}}

%% Defines the style of the programming listing
%% This is actually my personal template, go ahead and change stuff if you want



%% USEFUL LINKS:
%%
%%   UiO LaTeX guides:        https://www.mn.uio.no/ifi/tjenester/it/hjelp/latex/
%%   mathematics:             https://en.wikibooks.org/wiki/LaTeX/Mathematics

%%   PHYSICS !                https://mirror.hmc.edu/ctan/macros/latex/contrib/physics/physics.pdf

%%   the basics of Tikz:       https://en.wikibooks.org/wiki/LaTeX/PGF/Tikz
%%   all the colors!:          https://en.wikibooks.org/wiki/LaTeX/Colors
%%   how to draw tables:       https://en.wikibooks.org/wiki/LaTeX/Tables
%%   code listing styles:      https://en.wikibooks.org/wiki/LaTeX/Source_Code_Listings
%%   \includegraphics          https://en.wikibooks.org/wiki/LaTeX/Importing_Graphics
%%   learn more about figures  https://en.wikibooks.org/wiki/LaTeX/Floats,_Figures_and_Captions
%%   automagic bibliography:   https://en.wikibooks.org/wiki/LaTeX/Bibliography_Management  (this one is kinda difficult the first time)
%%   REVTeX Guide:             http://www.physics.csbsju.edu/370/papers/Journal_Style_Manuals/auguide4-1.pdf
%%
%%   (this document is of class "revtex4-1", the REVTeX Guide explains how the class works)


%% CREATING THE .pdf FILE USING LINUX IN THE TERMINAL
%%
%% [terminal]$ pdflatex template.tex
%%
%% Run the command twice, always.
%% If you want to use \footnote, you need to run these commands (IN THIS SPECIFIC ORDER)
%%
%% [terminal]$ pdflatex template.tex
%% [terminal]$ bibtex template
%% [terminal]$ pdflatex template.tex
%% [terminal]$ pdflatex template.tex
%%
%% Don't ask me why, I don't know.

\begin{document}

\title{FYS3150 oppgavesett 1}

\author{Jon Aleksander Prøitz and Marius Torsvoll}

\begin{abstract}
\noindent Relevant code can be found at:
\url{https://github.com/Jonaproitz/Project_1}
\end{abstract}

\maketitle
%\date{Received - / Accepted -}


\section*{Problem 1.}
    \label{sec:oppgave1}
    Given
    \begin{equation}
            u(x) 
        =   1-(1-e^{-10})x - e^{-10x}
        \label{u}
    \end{equation}
    Inserting $x = 0$ 
    \begin{equation*}
            u(0)
        =   1 - (1 - e^{-10}) \cdot 0 - e^{-10 \cdot 0}
        =   1 - 0 - 1
        =   0
    \end{equation*}
    and $x = 1$
    \begin{equation*}
            u(1)
        =   1 - (1 - e^{-10}) \cdot 1 - e^{-10 \cdot 1}
        =   1 - 1 + e^{-10} - e^{-10}
        =   0
    \end{equation*}
    furthermore the one-dimensional possion equation can be written
    \begin{equation*}
            -\frac{d^2 u}{d x^2} 
        =   -\frac{d^2}{d x^2}\left(1-(1-e^{-10})x - e^{-10x}\right)
        =   -\frac{d}{d x}\left((1-e^{-10}) + 10e^{-10x} \right) 
        =   100e^{-10x}
        =   f(x)
    \end{equation*}
    Hence \ref{u} is an exact solution to our problem.
    \hfill$\blacksquare$


\section*{Problem 2.}
    \label{sec:oppgave2}
    Se githublink 


\section*{Problem 3.}
    The one-dimensional poisson equation can be written
    \begin{equation*}
            -\frac{d ^2 u}{dx^2} 
        =   \frac{u(x-h)+2u(x)- u(x+h)}{h^2} + O(h^2)
        =   f(x)
    \end{equation*}
    Discretizing $x$ with a given distance $h$ between each distinct value then gives
    \begin{align*}
            x 
        \rightarrow 
            x_0, x_1, x_2, ..., x_m\\
            u(x) 
        \rightarrow 
            u_0, u_1, u_2, ..., u_m\\
            f(x) 
        \rightarrow 
            f_0, f_1 , f_2, ..., f_m
    \end{align*}
    with $u_i = v_i$, such that
    \begin{equation*}
            -\frac{d^2 v_i}{dx^2} 
        =   -v_{i-1} + 2v_i - v_{i+1} 
        =   f_ih^2
    \end{equation*}


\section*{Problem 4.}
    \label{sec:oppgave4}
    The set of equations from problem 3 can be written as 
    \begin{align*}
            -v_0 + 2v_1 - v_2 
        &=  h^2 f_1\\
            -v_1 + 2v_2 -v_3 
        &=  h^2 f_2\\
            -v_2 + 2v_3- v_4 
        &=  h^2 f_3\\
        \vdots&\\
            -v_{m-2} + 2 v_{m-1} - v_m
        &=  h^2 f_{m-1}
    \end{align*}
    Wich for 
    \begin{align*}
            g_1
        =&  h^2 f_1 + v_0\\
            g_2
        =&  h^2 f_2\\
            g_3
        =&  h^2 f_3\\
        &\vdots\\
            g_{m - 2}
        =&  h^2 f_{m-2}\\
            g_{m-1}
        =&  h^2 f_{m-1} + v_{m}
    \end{align*}
    can be written as the matrix equation
    \begin{equation*}
            \begin{pmatrix}
                2 & -1 & 0 &0 & 0 & \dots
                \\
                -1 & 2 & -1 & 0 & 0 & \dots
                \\
                0 & -1 & 2 & -1 & 0 & \dots
                \\
                \vdots & \vdots & & \ddots & \ddots
                \\
                0 & 0 & 0 & 0 & -1 & 2 
            \end{pmatrix}
            \begin{pmatrix}
                v_1 \\ v_2 \\ v_3 \\ \vdots \\ v_{m - 1} 
            \end{pmatrix} 
        =   \begin{pmatrix}
                g_1 \\ g_2 \\ g_3 \\ \vdots \\ g_{m - 1}
            \end{pmatrix}
    \end{equation*}


\section*{Problem 5}
    \subsection*{a}
        \label{sec:oppgave5a}
        When finding the matrix, $A$, in problem 4 it is assumed that $v_0$ and $v_m$ are known.
        Hence theese values are not calculated and
        \begin{equation*}
                n
            =   m - 2
        \end{equation*}
    \subsection*{b}
        \label{sec:oppgave5b}

\begin{acknowledgements}
Vi ønsker å takke det norske folk for deres støtte og hjelp i dette krevende arbeidet. 
\end{acknowledgements}

   
\end{document}
